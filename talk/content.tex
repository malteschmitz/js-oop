\usepackage{tikz}

\usepackage{listings}
\lstdefinelanguage{JavaScript}{
  keywords={typeof, new, true, false, catch, function, return, null, catch, switch, var, if, in, while, do, else, case, break},
  ndkeywords={class, export, boolean, throw, implements, import, this},
  sensitive=false,
  comment=[l]{//},
  morecomment=[s]{/*}{*/},
  morestring=[b]',
  morestring=[b]"
}

\lstset{%
  language=JavaScript,
  basicstyle=\ttfamily,%
  keywordstyle=\bfseries\color{yellow!70!black},%
  ndkeywordstyle=\color{blue!70!black},%
  stringstyle=\color{green!50!black},%
  commentstyle=\itshape\color{white!70!black},%
  showstringspaces=false,%
  upquote=true}
% german umlauts
\lstset{
  literate={ö}{{\"o}}1
           {ä}{{\"a}}1
           {ü}{{\"u}}1
           {ß}{{\ss}}1
} 

\title{Objektorientierte Programmierung mit JavaScript}

\author{Malte Schmitz}

\date{16. November 2012}

\begin{document}

\begin{frame}[fragile,plain]
  \begin{center}
    \LARGE Objektorientierte Programmierung \\ mit JavaScript 
  \end{center}
  
  \vskip4ex
  
  \begin{columns}
  \column{7.5em}
  \column{40em}
  \begin{lstlisting}[gobble=4]
    var meta = {
      author: "Malte Schmitz",
      date: new Date(2012, 10, 16),
      event: "MetaNook"
    }
  \end{lstlisting}
  \end{columns}
\end{frame}

\section*{Inhalt}

\begin{frame}{Ziele}
  \begin{enumerate}
    \item JavaScript-Objekte und -Funkionen verwenden können
    \item prototypische (differenzielle) Vererbung verstehen
    \item Objektorientierte Konzepte in JavaScript kennen lernen
    \item sehen, wie JavaScript-Frameworks diese umsetzen
  \end{enumerate}
\end{frame}

\begin{frame}{Gliederung}
  \begin{enumerate}[{Teil} I]
    \item Einführung in JavaScript
    \item Objektorientierung mit JavaScript
    \item Objektorientierung mit JavaScript-Frameworks
  \end{enumerate}
\end{frame}

\begin{frame}{Github}
  \url{github.com/malteschmitz/js-oop}
  \begin{itemize}
    \item diese Präsentation
    \item alle Beispiele
  \end{itemize}
\end{frame}

\section{JavaScript}

\subsection{Objekte}

\begin{frame}[fragile]{Objekte}
  \begin{itemize}
    \item \emph{keine} Instanz einer Klasse
    \item dynamische Sammlung von Eigenschaften (Hash, Map, \ldots)
  \end{itemize}
    
  \begin{lstlisting}[gobble=4]
    // Objekt erzeugen
    var o = {};
    
    // Eigenschaft setzen
    o.foo = 42;
    
    // Eigenschaft auslesen
    console.log(o.foo);
    
    // Eigenschaft löschen
    delete o.foo;
  \end{lstlisting}  
\end{frame}

\begin{frame}
  no content
\end{frame}

\subsection{Funktionen}

\begin{frame}
  no content
\end{frame}

\subsection{Funktionsaufrufe}

\begin{frame}
  no content
\end{frame}

\subsection{Vererbung}

\begin{frame}
  no content
\end{frame}

\section{Objektorientierung}

\subsection{pseudoklassische Vererbung}

\begin{frame}
  no content
\end{frame}

\subsection{prototypische Vererbung}

\begin{frame}
  no content
\end{frame}

\subsection{funktionale Vererbung}

\begin{frame}
  no content
\end{frame}

\subsection{Vererbung durch Kopieren}

\begin{frame}
  no content
\end{frame}

\section{Frameworks}

\begin{frame}
  no content
\end{frame}

\subsection{Underscore.js und \_.extend}

\begin{frame}
  no content
\end{frame}

\subsection{jQuery und \$.extend}

\begin{frame}
  no content
\end{frame}

\subsection{YUI und Y.extend}

\begin{frame}
  no content
\end{frame}

\subsection{Backbone.js und Backbone.Model.extend}

\begin{frame}
  no content
\end{frame}

\subsection{MooTools und new Class}

\begin{frame}
  no content
\end{frame}

\subsection{Prototype und Class.create}

\begin{frame}
  no content
\end{frame}

\end{document}