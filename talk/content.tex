\usepackage[utf8]{inputenc}

\usepackage[T1]{fontenc}

\usepackage{lmodern}
\usepackage{textcomp}
\usepackage{courier}

\usepackage[ngerman]{babel}
\usepackage[german=guillemets]{csquotes}

\usefonttheme{structurebold}
\setbeamertemplate{navigation symbols}{} % hide navigation symbols
\setbeamertemplate{footline}{
  \usebeamercolor[fg]{page number in head/foot}%
  \usebeamerfont{page number in head/foot}%
  \centering--\,\insertframenumber\,--\vskip1em%
}
\setbeamertemplate{sections/subsections in toc}[sections numbered]

\setbeamercolor{page number in head/foot}{parent=normal text}

\AtBeginSection[]{
  \begin{frame}
    \Huge \insertsection
    
    \large \vskip1em
    \tableofcontents[sectionstyle=hide/hide,subsectionstyle=show/show/hide]
  \end{frame}
}

\usepackage{tikz}
\usetikzlibrary{shapes.multipart,positioning}

\usepackage{listings}
\lstdefinelanguage{JavaScript}{
  keywords={typeof, new, true, false, catch, function, return, null, catch, switch, var, if, in, while, do, else, case, break, for},
  ndkeywords={boolean, throw, this},
  sensitive=false,
  comment=[l]{//},
  morecomment=[s]{/*}{*/},
  morestring=[b]',
  morestring=[b]"
}

\lstset{%
  language=JavaScript,
  basicstyle=\ttfamily,%
  keywordstyle=\bfseries\color{keywordcolor},%
  ndkeywordstyle=\color{ndkeywordcolor},%
  stringstyle=\color{stringcolor},%
  commentstyle=\itshape\color{commentcolor},%
  showstringspaces=false,%
  upquote=true}
% german umlauts
\lstset{
  literate={ö}{{\"o}}1
           {ä}{{\"a}}1
           {ü}{{\"u}}1
           {ß}{{\ss}}1
} 

\title{Objektorientierte Programmierung mit JavaScript}

\author{Malte Schmitz}

\date{16. November 2012}

\begin{document}

\begin{frame}[fragile,plain]
  \begin{center}
    \LARGE Objektorientierte Programmierung \\ mit JavaScript 
  \end{center}
  
  \vskip4ex
  
  \begin{columns}
  \column{7.5em}
  \column{40em}
  \begin{lstlisting}[gobble=4]
    var meta = {
      author: 'Malte Schmitz',
      date: new Date(2012, 10, 16),
      event: 'MetaNook'
    }
  \end{lstlisting}
  \end{columns}
\end{frame}

\section*{Inhalt}

\begin{frame}{Ziele}
  \begin{enumerate}
    \item JavaScript-Objekte und -Funkionen verwenden können
    \item prototypische (differenzielle) Vererbung verstehen
    \item Objektorientierte Konzepte in JavaScript kennen lernen
    \item sehen, wie JavaScript-Frameworks diese umsetzen
  \end{enumerate}
\end{frame}

\begin{frame}{Gliederung}
  \begin{enumerate}[{Teil} I]
    \item Einführung in JavaScript
    \item Objektorientierung mit JavaScript
    \item Objektorientierung mit JavaScript-Frameworks
  \end{enumerate}
\end{frame}

\begin{frame}{Github}
  \url{github.com/malteschmitz/js-oop}
  \begin{itemize}
    \item diese Präsentation
    \item \LaTeX-Quellcode
    \item JavaScript-Beispiele
  \end{itemize}
\end{frame}

\section{JavaScript}

\subsection{Objekte}

\begin{frame}[fragile]{Variablen}
  \begin{lstlisting}[gobble=4]
    // Deklaration
    var a;
    console.log(a); // --> undefined
    
    // Initialisierung
    a = 42;
    console.log(a); // --> 42
  \end{lstlisting}
  
  \begin{itemize}
    \item dynamische Typisierung, \emph{nicht} ungetypt
    \item Sichtbarkeitsbereich: ganze aktuelle Funktion
  \end{itemize}
  
  \begin{lstlisting}[gobble=4]
    console.log(b); // --> undefined
    // Deklaration und Initialisierung
    var b = 23; 
  \end{lstlisting}  
\end{frame}

\begin{frame}[fragile]{Kontrollstrukturen}
  \begin{lstlisting}[gobble=4]
    var x = 13;
    while (x < 15) {
      console.log(x); // --> 13, 14
      x += 1;
    }
    if (x >= 15) {
      console.log(x); // --> 15
    }
  \end{lstlisting}
\end{frame}

\begin{frame}[fragile]{Objekte}
  \begin{itemize}
    \item \emph{keine} Instanz einer Klasse
    \item dynamische Sammlung von Eigenschaften (Hash, Map, \ldots)
    \item alles ist ein Objekt
  \end{itemize}
  
  \begin{onlyenv}<1>
    \begin{lstlisting}[gobble=6]
      // Objekt erzeugen
      var o = {};
      
      // Eigenschaft setzen
      o.foo = 42;
      
      // Eigenschaft auslesen
      console.log(o.foo); // --> 42
      
      // Eigenschaft löschen
      delete o.foo;
    \end{lstlisting}
  \end{onlyenv}
  
  \begin{onlyenv}<2>
    \begin{lstlisting}[gobble=6]
      // Objekt erzeugen
      var o = {};
      
      // Eigenschaft setzen
      o['foo'] = 42;
      
      // Eigenschaft auslesen
      console.log(o['foo']); // --> 42
      
      // Eigenschaft löschen
      delete o['foo'];
    \end{lstlisting}
  \end{onlyenv}
\end{frame}

\begin{frame}[fragile]{Objektliterale}
  \begin{lstlisting}[gobble=4]
    var cat = {
      name: 'Micia',
      age: 12
    }
  \end{lstlisting}
\end{frame}

\begin{frame}[fragile]{Iteration über Objekteigenschaften}
  \begin{lstlisting}[gobble=4]
    for (var key in meta) {
      var value = meta[key];
      console.log(key + ': ' + value);
    }
  \end{lstlisting}
\end{frame}

\begin{frame}[fragile]{\texttt{\&\&} und \texttt{||}}
  \begin{lstlisting}[gobble=4]
    // guard operator
    var author = meta && meta.author;
    
    // default operator
    var obj = meta || {};
  \end{lstlisting}
\end{frame}

\begin{frame}[fragile]{Duck-Typing}
  \begin{quotation}
    When I see a bird that walks like a duck and swims like a duck and
    quacks like a duck, I call that bird a duck.
    
    \raggedleft -- James Whitcomb Riley
  \end{quotation}
  
  \vskip5ex
  
  \begin{lstlisting}[gobble=4]
    if (meta.author) {
      console.log(meta.author);
    }
  \end{lstlisting}
\end{frame}

\begin{frame}[fragile]{globale Variablen}
  \begin{itemize}
    \item Eigenschaften des globalen Objektes
    \item \lstinline-window- im Browser
  \end{itemize}
  
  \vskip3ex
  
  \begin{lstlisting}[gobble=4]
    foo = 12;
    // entspricht
    window.foo = 12;
  \end{lstlisting}
\end{frame}

\subsection{Vererbung}

\begin{frame}{differenzielle Vererbung}
  \begin{block}{Prototyp}
    Prototyp ist ein GoF-Entwurfsmuster. Neue Instanzen\\
    basieren auf prototypischen Instanzen (\enquote{Vorlagen}).
  \end{block}

  \vskip3ex

  \begin{itemize}
    \item zwischen Objekten; \emph{keine} Klassen
    \item versteckter Verweis zum Prototypen
    \item beim Lesen Rückgriff auf Prototypen
    \item alle Objekte erben von \lstinline-Object.prototype-  
  \end{itemize}
\end{frame}  
  
\begin{frame}[fragile]{differenzielle Vererbung}
  \begin{tikzpicture}[shorten >=1pt]
    \uncover<4->{
      \node[inner sep=0pt] (cat) {
        \begin{tabular}{|l|l|}
          \hline
          proto & \\ \hline
          \uncover<6-7>{\lstinline-name-} & \uncover<6-7>{\lstinline-Micia-} \\ \hline 
        \end{tabular}
      };
      \node[anchor=south west,inner sep=1pt] at (cat.north west) {\lstinline-cat-};
      \node[circle,draw,fill,inner sep=2pt,xshift=2em,yshift=1.5ex] at (cat.center) (cat dot) {};
    }
    \uncover<2->{
      \node[inner sep=0pt,node distance=1em,right=of cat] (dog) {
        \begin{tabular}{|l|l|}
          \hline
          proto & \\ \hline
          \lstinline-name- & \lstinline-Cody- \\ \hline 
        \end{tabular}
      };
      \node[anchor=south west,inner sep=1pt] at (dog.north west) {\lstinline-dog-};
      \node[circle,draw,fill,inner sep=2pt,xshift=2em,yshift=1.5ex] at (dog.center) (dog dot) {};
      \node[right=of dog dot] (obj) {\lstinline-Object.prototype-};
    }
    \only<4->{\draw[->] (cat dot) edge (dog.west |- cat dot);}
    \only<2->{\draw[->] (dog dot) edge (obj);}
  \end{tikzpicture}
  
  \vskip3ex
  
  \uncover<1->{\lstinline-var dog = \{name: 'Cody'\};-}\\
  \uncover<3->{\lstinline-var cat = Object.create(dog);-}\\
  \uncover<4->{\lstinline|console.log(cat.name); // --> Cody|}
  \uncover<5->{\lstinline-cat.name = 'Micia';-}\\
  \uncover<6->{\lstinline|console.log(cat.name); // --> Micia|}
  \uncover<7->{\lstinline-delete cat.name;-}
  \uncover<8->{\lstinline|console.log(cat.name); // --> Cody|}
\end{frame}

\subsection{Funktionen}

\begin{frame}[fragile]{Funktionen}
  \begin{lstlisting}[gobble=4]
    var square = function(x) {
      return x * x;
    }
  \end{lstlisting}
\end{frame}

\begin{frame}[fragile]{Funktionen}
  \begin{lstlisting}[gobble=4]
    var sum = function() {
      var s = 0, i = 0;
      for (;i < arguments.length; i++) {
        s += arguments[i];
      }
      return s;
    }
  \end{lstlisting}
\end{frame}

\begin{frame}[fragile]{Funktionsabschluss}
  \begin{lstlisting}[gobble=4]
    var makeIncrementer = function(n) {
      return function() {
        return n++;
      }
    }
    
    a = makeIncrementer(1);
    b = makeIncrementer(5);
    
    console.log(a()); // --> 1
    console.log(b()); // --> 5
    console.log(a()); // --> 2
    console.log(b()); // --> 6
  \end{lstlisting}
\end{frame}

\subsection{Funktionsaufrufe}

\begin{frame}{Funktionsaufrufe}
  \begin{tabular}{ll}
    Aufruf & \lstinline-this- \\ \hline
    \lstinline-sum()- & globales Object \\
    \lstinline-obj.sum()- & \lstinline-obj- \\
    \lstinline-sum.apply(foo)- & \lstinline-foo- \\
    \lstinline-new sum()- & neues Object
  \end{tabular}
\end{frame}

\begin{frame}[fragile]{Funktionsaufrufe}
  \begin{lstlisting}[gobble=4]
    sum(1,2,3);
    sum.apply(window, [1, 2, 3]);
    sum.call(window, 1, 2, 3);
    
    
    meta.add = sum;
    
    meta.add(1,2,3);
    sum.apply(meta, [1, 2, 3]);
    sum.call(meta, 1, 2, 3);
  \end{lstlisting}
\end{frame}

\begin{frame}[fragile]{\lstinline-prototype--Eigenschaft}
  \begin{lstlisting}[gobble=4]
    var f = function() {};
    
    // impliziert
    
    f.prototype = {
      constructor: f
    };
  \end{lstlisting}
\end{frame}

\begin{frame}[fragile]{\lstinline-new--Operator}
  \begin{lstlisting}[gobble=4]
    var a = new Fun(42);

    // entspricht  
  
    var a = Object.create(Fun.prototype);
    var res = Fun.call(a, 42);
    if (res && (typeof res === 'object' ||
                typeof res === 'function')) {
      a = res;
    }
  \end{lstlisting}
\end{frame}

\begin{frame}[fragile]{\lstinline-new--Operator}
  \begin{lstlisting}[gobble=4]
    var Animal = function(name) {
      this.name = name;
    };
    Animal.prototype.say = function(text) {
      console.log(this.name + ': ' + text)
    };
    
    var erwin = new Animal('Erwin');
    erwin.say('Hello'); // --> Erwin: Hello
  \end{lstlisting}
\end{frame}

\begin{frame}[fragile]{Iteration über \emph{eigene} Objekteigenschaften}
  \begin{lstlisting}[gobble=4]
    for (var key in meta) {
      if (Object.prototype.
          hasOwnProperty.apply(meta, key)) {
        var value = meta[key];
        console.log(key + ': ' + value);
      }
    }
  \end{lstlisting}
\end{frame}

\begin{frame}[fragile]{\lstinline-Object.create-}
  \begin{lstlisting}[gobble=4]
    if (typeof Object.create !== 'function') {
      Object.create = function(obj) {
        var F = function(){};
        F.prototype = obj;
        return new F();
      }
    }
  \end{lstlisting}  
\end{frame}

\section{Objektorientierung}

\newcommand{\hla}[1]{\textcolor{stringcolor}{#1}}
\newcommand{\hlb}[1]{\textcolor{keywordcolor}{#1}}

\begin{frame}{Objektorientierung}
  \begin{description}
    \item[\hla{Verweis}] versteckter Verweis zum Prototypen \\
    \item[\hlb{Kopie}] Kopien aller Eigenschaften des Prototyps
  \end{description}
  
  \vskip3ex
  
  \begin{itemize}
    \item \hla{direkt-prototypische Vererbung (Verweis)}
    \item \hlb{direkt-kopierende Vererbung (Kopie)}
    \item \hla{pseudoklassische Vererbung (Verweis)}
    \item \hla{prototypische Vererbung (Verweis)}
    \item \hlb{funktionale Vererbung (Kopie)}
  \end{itemize}
\end{frame}

\subsection{direkt-prototypische Vererbung}

\begin{frame}[fragile]{direkt-prototypische Vererbung}
  \begin{lstlisting}[gobble=4]
    var animal = {
      toString: function() {
        return this.name;
      },
      name: 'animal'
    };
    
    
    var cat = Object.create(animal);
    cat.name = 'Micia';
    cat.meow = function() {
      console.log(this.name + ': Meow!');
    }
    console.log(cat); // --> Micia
  \end{lstlisting}
\end{frame}

\subsection{direkt-kopierende Vererbung}

\begin{frame}[fragile]{direkt-kopierende Vererbung}
  \begin{lstlisting}[gobble=4]
    var animal = {
      toString: function() {
        return this.name;
      },
      name: 'animal'
    };
      
    var cat = {};
    copyProperties(animal, cat);
    cat.name = 'Micia';
    cat.meow = function() {
      console.log(this.name + ': Meow!');
    }
    console.log(cat); // --> Micia
  \end{lstlisting}
\end{frame}

\begin{frame}[fragile]{direkt-kopierende Vererbung}
  \begin{lstlisting}[gobble=4]
    var ownProp = Object.prototype.hasOwnProperty;
  
    var copyProperties = function(from,to) {
      for (var key in from) {
        if (ownProp.call(from, key)) {
          to[key] = from[key];
        }
      }
    }
  \end{lstlisting}
\end{frame}

\subsection{pseudoklassische Vererbung}

\lstset{basicstyle=\ttfamily\scriptsize}

\begin{frame}[fragile]{pseudoklassische Vererbung}
  \begin{lstlisting}[gobble=4]
    var Animal = function(name) {
      this.name = name.toUpperCase();
    }
    Animal.prototype.toString = function() {
      return this.name;
    }
    
    var Cat = function(name) {
      this.name = name;
    }
    Cat.prototype = new Animal();
    Cat.prototype.constructor = Cat;
    Cat.prototype.meow = function() {
      console.log(this.name + ': Meow!');
    }
    
    var cat = new Cat('Micia');
    console.log(cat); // --> MICIA
  \end{lstlisting}
\end{frame}

\subsection{prototypische Vererbung}

\begin{frame}[fragile]{prototypische Vererbung}
  \begin{lstlisting}[gobble=4]
    var animal = makeConstructor(Object, function(name) {
      this.name = name.toUpperCase();
    }, {
      toString: function() {
        return this.name;
      }
    });
    
    var cat = makeConstructor(animal, function(name) {
      this.name = name;
    }, {
      meow: function() {
        console.log(this.name + ': Meow!');
      }
    });
    
    var c = cat('Micia');
    console.log(c); // --> MICIA
  \end{lstlisting}
\end{frame}

\begin{frame}[fragile]{prototypische Vererbung}
  \begin{lstlisting}[gobble=4]
    var makeConstructor = function(extend, initializer, methods) {
      var func = function() {
        var that = Object.create(proto);
        if (typeof initializer === 'function') {
          initializer.apply(that, arguments);
        }
        return that;
      }
      var proto = Object.create(extend && extend.prototype);
      copyProperties(methods, proto);
      func.prototype = proto;
      proto.constructor = func;
      return func;
    }
  \end{lstlisting}
\end{frame}

\subsection{funktionale Vererbung}

\begin{frame}[fragile]{funktionale Vererbung}
  \begin{lstlisting}[gobble=4]
    var animal = function(name) {
      return {
        name: name.toUpperCase(),
        toString: function() {
          return this.name;
        }
      }
    }
    
    var cat = function(name) {
      var that = animal(name);
      that.meow = function() {
        console.log(this.name + ': Meow!');
      }
      return that;
    }
    
    var c = cat('Micia');
    console.log(c); // --> MICIA
  \end{lstlisting}
\end{frame}

\begin{frame}[fragile]{funktionale Vererbung}
  \begin{lstlisting}[gobble=4]
    var animal = function(name) {
      name = name.toUpperCase();
      return {
        toString: function() {
          return name;
        }
      }
    }
    
    var cat = function(name) {
      var that = animal(name);
      that.meow = function() {
        console.log('???' + ': Meow!');
      }
      return that;
    }
    
    var c = cat('Micia');
    console.log(c); // --> MICIA
  \end{lstlisting}
\end{frame}

\begin{frame}[fragile]{funktionale Vererbung}
  \begin{lstlisting}[gobble=4]
    var animal = function(name, secret) {
      secret = secret || {};
      secret.name = name.toUpperCase();
      return {
        toString: function() {
          return secret.name;
        }
      }
    }
    
    var cat = function(name) {
      var secret = {};
      var that = animal(name, secret);
      that.meow = function() {
        console.log(secret.name + ': Meow!');
      }
      return that;
    }
    
    var c = cat('Micia');
    console.log(c); // --> MICIA
  \end{lstlisting}
\end{frame}

\section{Frameworks}

\subsection{Underscore.js und \texttt{\_.extend}}

\begin{frame}[fragile]{Underscore.js und \texttt{\_.extend}}
  \begin{lstlisting}[basicstyle=\ttfamily,gobble=4]
    var person = {name: 'Malte'};
    var contact = {mail: 'm@mlte.de'};
    var data = {age: 23, sex: 'male'};
    // extend person with concat and data
    _.extend(person, contact, data);
  \end{lstlisting}
\end{frame}

\begin{frame}[fragile]{Underscore.js und \texttt{\_.extend}}
  \begin{lstlisting}[basicstyle=\ttfamily,gobble=4]
    var _ = {};
    _.extend = function(obj) {
      var sources = makeArray(arguments).slice(1);
      for (var key in sources) {
        var source = sources[key];
        for (var prop in source) {
          obj[prop] = source[prop];
        }
      }
      return obj;
    }
  \end{lstlisting}
\end{frame}

\subsection{jQuery und \texttt{\$.extend}}

\begin{frame}[fragile]{jQuery und \texttt{\$.extend}}
  \begin{lstlisting}[gobble=4]
    // shallow copy
    var person = {name: 'Malte'};
    var meta = {options: {say: 'Hello'}};
    $.extend(false, person, meta);
    person.options.say = 'Good Bye';
    console.log(meta.options.say); // --> Good Bye
    
    // deep copy
    person = {name: 'Malte'};
    meta = {options: {say: 'Hello'}};
    $.extend(true, person, meta);
    person.options.note = 'Good Bye';
    console.log(meta.options.say); // --> Hello
  \end{lstlisting}
\end{frame}

\begin{frame}[fragile]{jQuery und \texttt{\$.extend}}
  \begin{lstlisting}[gobble=4]
    var jQuery = {}, $=jQuery;
    
    jQuery.extend = function(deep) {
      var options, name, src, copy, clone, i,
        target = arguments[1] || {};    
      for (i = 1; i < arguments.length; i++) {
        if ((options = arguments[i]) != null) {
          for (name in options) {
            src = target[name];
            copy = options[name];
            if (deep && copy && jQuery.isPlainObject(copy)) {
              clone = jQuery.isPlainObject(src) ? src : {};
              target[name] = jQuery.extend(deep, clone, copy);
            } else if (copy !== undefined) {
              target[name] = copy;
            }
          }
        }
      }
      return target;
    };
  \end{lstlisting}
\end{frame}

\begin{frame}[fragile]{\texttt{typeof} ist kaputt}
  \begin{tabular}{ll}
    Typ & \lstinline-typeof- \\ \hline
    \lstinline-undefined- & \lstinline-'undefined'- \\
    \lstinline-null- & \lstinline-'object'- \\
    Array & \lstinline-'object'- \\
    Boolean & \lstinline-'boolean'- \\
    Number & \lstinline-'number'- \\
    String & \lstinline-'string'- \\
    Funktion & \lstinline-'function'- \\
    alles andere & \lstinline-'object'-
  \end{tabular}
\end{frame}

\begin{frame}[fragile]{jQuery und \texttt{\$.type}}
  \begin{lstlisting}[gobble=4]
    var class2type = [];
    var types = ['Boolean', 'Number', 'String', 'Function',
      'Array', 'Date', 'RegExp', 'Object'];
    for (var i = 0; i < types.length; i++) {
      class2type['[object ' + types[i] + ']'] = types[i].toLowerCase();
    }
    jQuery.type = function(obj) {
      return obj == null ?
        String(obj) :
        class2type[Object.prototype.toString.call(obj)] || 'object';
    };
  \end{lstlisting}
\end{frame}

\begin{frame}[fragile]{jQuery und \texttt{\$.isPlainObject}}
  \begin{lstlisting}[gobble=4]
    jQuery.isPlainObject = function(obj) {
      var ownProp = Object.prototype.hasOwnProperty;
    
      if (!obj || jQuery.type(obj) !== 'object' ||
          obj.nodeType || obj == obj.window) {
        return false;
      }
    
      if (obj.constructor &&
        !ownProp.call(obj, 'constructor') &&
        !ownProp.call(obj.constructor.prototype, 'isPrototypeOf')) {
        return false;
      }
    
      // Own properties are enumerated firstly, so to speed up,
      // if last one is own, then all properties are own.
    
      var key;
      for (key in obj) {}
    
      return key === undefined ||
        ownProp.call(obj, key);
    };
  \end{lstlisting}
\end{frame}

\subsection{YUI und \texttt{Y.extend}}

\begin{frame}[fragile]{YUI und \texttt{Y.extend}}
  \begin{lstlisting}[gobble=4]
    var Person = function() {
      this.name = 'Adam';
    }
    Person.prototype.getName = function() {
      return this.name;
    };
    var Programmer = function(){}
    Y.extend(Programmer, // the object to modify
      Person, // the object to inherit
      {foo: true}, // prototype properties to add/override
      {limit: 'sky'} // static properties to add/override
    );
    
    var guru = new Programmer();
    console.log(guru.getName()); // --> Adam
    console.log(guru.foo); // --> true
    console.log(Programmer.limit); // --> Sky
  \end{lstlisting}
\end{frame}

\begin{frame}[fragile]{YUI und \texttt{Y.extend}}
  \begin{lstlisting}[gobble=4]
    var Y = {};    
    // r - receiver: the function to modify
    // s - sender: the function to inherit from
    // px - prototype properties to add to r.prototype
    // sx - static propoperties to add to r
    Y.extend = function(r, s, px, sx) {
      r.prototype = Object.create(s.prototype);
      r.prototype.constructor = r;
      // fix constructor property of sender
      if (s != Object && s.prototype.constructor ==
                  Object.prototype.constructor) {
          s.prototype.constructor = s;
      }
      // add prototype overrides
      if (px) {
          copyProperties(px, r.prototype);
      }
      // add object overrides
      if (sx) {
          copyProperties(sx, r);
      }
      return r;
    };
  \end{lstlisting}
\end{frame}

\subsection{Backbone.js und \texttt{Backbone.Model.extend}}

\begin{frame}[fragile]{Backbone.js und \texttt{Backbone.Model.extend}}
  \begin{lstlisting}[gobble=4]
    var FooModel = Backbone.Model.extend({
      f: function() {
        return 'instance foo';
      }
    }, {
      f: function() {
        return 'static foo'
      }
    });
    
    var BarModel = FooModel.extend({
      f: function() {
        return 'instance bar'
      }
    }, {
      f: function() {
        return 'static bar'
      }
    });
    
    var foo = new FooModel();
    var bar = new BarModel();
  \end{lstlisting}
\end{frame}

\begin{frame}[fragile]{Backbone.js und \texttt{Backbone.Model.extend}}
  \begin{lstlisting}[gobble=4]
    var Backbone = {
      Model: function() {}
    };
    
    Backbone.Model.extend = function(protoProps, classProps) {
      var child = inherits(this, protoProps, classProps);
      child.extend = this.extend;
      return child;
    };
  \end{lstlisting}
\end{frame}

\begin{frame}[fragile]{Backbone.js und \texttt{Backbone.Model.extend}}
  \begin{lstlisting}[gobble=4]
    var inherits = function(parent, protoProps, staticProps) {
      var child;
    
      if (protoProps && protoProps.hasOwnProperty('constructor')) {
        child = protoProps.constructor;
      } else {
        child = function(){ parent.apply(this, arguments); };
      }
    
      // Inherit class (static) properties from parent.
      _.extend(child, parent);
    
      child.prototype = Object.create(parent.prototype);
      if (protoProps) _.extend(child.prototype, protoProps);
      if (staticProps) _.extend(child, staticProps);
      child.prototype.constructor = child;
    
      return child;
    };
  \end{lstlisting}
\end{frame}

\subsection{MooTools und \texttt{new Class}}

\begin{frame}[fragile]{MooTools und \texttt{new Class}}
  \begin{lstlisting}[gobble=4]
    var Animal = new Class({
      initialize: function(age){
        this.age = age;
      }
    });
    var Cat = new Class({
      Extends: Animal,
      setName: function(name){
        this.name = name
      }
    });
    var Dog = new Class({
      Implements: Animal,
      setName: function(name){
        this.name = name
      }
    });
    var cat = new Cat(20);
    var dog = new Dog(7);
  \end{lstlisting}
\end{frame}

\begin{frame}[fragile]{MooTools und \texttt{new Class}}
  \begin{lstlisting}[gobble=4]
    var Class = function(params){
      if (typeof params === 'function') {
        params = {initialize: params};
      }
      var newClass = function(){
        reset(this);
        if (this.initialize) {
          return this.initialize.apply(this, arguments);
        }
      }
      copyProperties(this,newClass);
      implementProperties(params,newClass);
      return newClass;
    };
    
    Class.Mutators = {
      Extends: function(parent){
        this.prototype = Object.create(parent.prototype);
      },
      Implements: function(item){
        implementProperties(item, this);
      }
    };
  \end{lstlisting}
\end{frame}

\begin{frame}[fragile]{MooTools und \texttt{new Class}}
  \begin{lstlisting}[gobble=4]
    var implementProperties = function(from,to) {
      for (var key in from) {
        if (Object.prototype.hasOwnProperty.call(from, key)) {
          if (Class.Mutators.hasOwnProperty(key)){
            Class.Mutators[key].call(to, from[key]);
          } else {
            to.prototype[key] = from[key];
          }
        }
      }
    }
    
    var reset = function(object){
    	for (var key in object){
    		var value = object[key]; 
    		if (typeof value === 'object') {
          object[key] = reset(Object.create(value));
        }
    	}
    	return object;
    };
  \end{lstlisting}
\end{frame}

\subsection{Prototype und \texttt{Class.create}}

\begin{frame}[fragile]{Prototype und \texttt{Class.create}}
  \begin{lstlisting}[gobble=4]
    var Person = Class.create({
      initialize: function(name) {
        this.name = name;
      },
      say: function(message) {
        return this.name + ': ' + message;
      }
    });
    
    var Pirate = Class.create(Person, {
      say: function($super, message) {
        return $super(message) + ', yarr!';
      }
    });
    
    var john = new Pirate('Long John');
    console.log(john.say('ahoy matey'));
        // --> Long John: ahoy matey, yarr!
  \end{lstlisting}
\end{frame}

\begin{frame}[fragile]{Prototype und \texttt{Class.create}}
  \begin{lstlisting}[gobble=4]
    var Class = {Methods: {}};
    Class.create = function() {
      var parent = null, properties = makeArray(arguments);
      if (typeof properties[0] === 'function')
        parent = properties.shift();
  
      var klass = function() {
        this.initialize.apply(this, arguments);
      }  
      copyProperties(Class.Methods, klass);
      if (parent) {
        klass.prototype = Object.create(parent.prototype)
      }
      klass.prototype.constructor = klass;
  
      for (var i = 0; i < properties.length; i++)
        klass.addMethods(properties[i]);
  
      if (!klass.prototype.initialize)
        klass.prototype.initialize = function() {};
  
      return klass;
    }
  \end{lstlisting}
\end{frame}

\begin{frame}[fragile]{Prototype und \texttt{Class.create}}
  \begin{lstlisting}[gobble=4]
    Class.Methods.addMethods = function(source) {
      var ancestor = this.superclass && this.superclass.prototype;
  
      for (var property in source) {
        var value = source[property];
        if (ancestor && typeof value === 'function' &&
            argumentNames(value)[0] === '$super') {
          value = (function(p,v) {
            return function() {
              var a = [ancestor[p].bind(this)].
                concat(makeArray(arguments));
              return v.apply(this, a);
            };
          })(property, value);
        }
        this.prototype[property] = value;
      }
  
      return this;
    }
  \end{lstlisting}
\end{frame}

\begin{frame}[fragile]{Prototype und \texttt{Class.create}}
  \begin{lstlisting}[gobble=4]
    var argumentNames = function(f) {
      var re = /^[\s\(]*function[^(]*\(([^)]*)\)/;
      var names = f.toString().match(re)[1]
        .replace(/\/\/.*?[\r\n]|\/\*(?:.|[\r\n])*?\*\//g, '')
        .replace(/\s+/g, '').split(',');
      return names.length == 1 && !names[0] ? [] : names;
    }
  \end{lstlisting}
\end{frame}

\section*{Zusammenfassung}

\begin{frame}{Zusammenfassung}
  \begin{itemize}
    \item \alert{Prototypische/differenzielle Vererbung} ist eine \alert{Verallgemeinerung} der
      klassenbasierten Verberbung.
    \item \alert{Klassenbasierte Vererbung} kann in JavaScript \alert{simuliert} werden.
    \item Es muss zwischen \alert{prototypischer Vererbung} und \alert{Vererbung durch
      Kopieren} unterschieden werden.
    \item In der Praxis sucht man sich eines der \alert{vielen Frameworks} für
      \alert{Objektorientierte Programmierung mit JavaScript}.
  \end{itemize} 
\end{frame}

\begin{frame}{Crockford on JavaScript}
  \begin{thebibliography}{1}
    \bibitem{1} Volume One:
    \emph{The Early Years}
    \newblock 25. Januar 2010
    
    \bibitem{2} Chapter 2:
    \emph{And Then There Was JavaScript}
    \newblock 5. Februar 2010
    
    \bibitem{3} Act III:
    \emph{Function the Ultimate}
    \newblock 17. Februar 2010
    
    \bibitem{4} Episode IV:
    \emph{The Metamorphosis of Ajax}
    \newblock 3. März 2010
    
    \bibitem{5} Part 5:
    \emph{The End of All Things}
    \newblock 31. März 2010
    
    %\bibitem{6} Scene 6:
    %\emph{Loopage}
    %\newblock 27. August 2010
    %
    %\bibitem{7} Level 7:
    %\emph{ECMAScript 5: The New Parts}
    %\newblock 29. März 2011
    %
    %\bibitem{8} Section 8:
    %\emph{Programming Style and Your Brain}
    %\newblock 3. November 2011
  \end{thebibliography}
\end{frame}

\begin{frame}{Bücher}
  \begin{thebibliography}{1}
    \bibitem{1} David Flanagan
    \newblock \emph{JavaScript: The Definitive Guide}
    
    \bibitem{2} Douglas Crockford
    \newblock \emph{JavaScript: The Good Parts}
  \end{thebibliography}
\end{frame}

\appendix

\section{Anhang}

\subsection{partielle Applikation}

\begin{frame}[fragile]{\lstinline-makeArray-}
  \begin{lstlisting}[gobble=4]
    var makeArray = function(a) {
      return Array.prototype.slice.call(a);
    }
  \end{lstlisting}  
\end{frame}

\begin{frame}[fragile]{partielle Applikation}
  \begin{onlyenv}<1>
    \begin{lstlisting}[gobble=6]
      var add = function(a,b) {
        return a + b;
      }
      var inc = partial(add,1);
      console.log(inc(6)); // --> 7
    \end{lstlisting}
  \end{onlyenv}
  
  \begin{onlyenv}<2>
    \begin{lstlisting}[gobble=6]
      var partial = function(func) {
        var fixed = makeArray(arguments).slice(1);
        return function() {
          var args = makeArray(arguments);
          return func.apply(null, fixed.concat(args));
        }
      }
    \end{lstlisting}
  \end{onlyenv}
\end{frame}

\end{document}